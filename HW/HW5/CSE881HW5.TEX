\documentclass[11pt]{scrartcl}
\usepackage[utf8]{inputenc}
\usepackage{mathtools}
\usepackage{amssymb}
\usepackage{listings}
\usepackage{bm}
\usepackage{graphicx}
\usepackage{xcolor}
\lstset
{ %Formatting for code in appendix
	language=Matlab,
	basicstyle=\footnotesize,
	numbers=left,
	stepnumber=1,
	showstringspaces=false,
	tabsize=1,
	breaklines=true,
	breakatwhitespace=false,
}
\begin{document}
\centerline{\LARGE{\textbf{CSE881 HW5}}}
\centerline{\large{\textit{Nan Cao,\  A52871775}}}
\centerline{\large{\textit{Oct 08th, 2016}}}
\section*{Problem 1}
\textbf{(a)}
\begin{center}
\begin{tabular}{|c|c|c|c|}
	\hline
 \multicolumn{2}{|c|}{Tree 1}
 &\multicolumn{2}{|c|}{Predicted}\\
 \cline{3-4}
 \multicolumn{2}{|c|}{}& +  & -  \\
 \cline{1-4}
	Actual & +         & 20 & 10 \\
 \cline{2-4}
	       & -         & 10 & 20\\
 \hline
\end{tabular}
\ \ \ \ \ \ \ \ \ \ \ \ \ \ \ 
%\end{center}
%\begin{center}
\begin{tabular}{|c|c|c|c|}
 \hline
 \multicolumn{2}{|c|}{Tree 2}
&\multicolumn{2}{|c|}{Predicted}\\
 \cline{3-4}
 \multicolumn{2}{|c|}{}& +  & -  \\
 \cline{1-4}
		        Actual & +  & 25 & 5 \\
 \cline{2-4}
		               & -  & 20 & 10  \\
 \hline
\end{tabular}
\end{center}
\textbf{(b)}
\begin{equation*}
\begin{aligned}
error_1&=\frac{10+10}{60}=\frac{1}{3}\\
error_2&=\frac{5+20}{60}=\frac{5}{12}\\
\end{aligned}
\end{equation*}
Tree 1 has a lower training error.\\
\textbf{(c)}
\begin{equation*}
\begin{aligned}
Cost(
&= Cost(Tata|tree)+Cost(Tree)\\
Cost_1&=3[log_{2}4]+4[log_{2}2]+20[log_{2}60]\\
&=6+4+120=130bits\\
Cost_2&=2[log_{2}4]+3[log_{2}2]25[log_{2}60]\\
&=4+3+180=157bits\\
\end{aligned}
\end{equation*}
So Tree 1 is better.\\
\section*{Problem 2}
\textbf{(a)}
Yes\\
\begin{equation*}
\begin{aligned}
\phi_1 (\bm{x})&=x_1 x_2\\
\phi_2 (\bm{x})&=x_3 x_4\\
\bm{w}&=(1,-1)\\
f(\bm{x})&=x_1 x_2 - x_3 x_4\\
\end{aligned}
\end{equation*}
\textbf{(b)}
No.\\
%\begin{equation*}
%\begin{aligned}
%\end{aligned}
%\end{equation*}
\textbf{(c)}
No\\
%Yes\\
%\begin{equation*}
%\begin{aligned}
%\phi_1 (\bm{x})&=x_1\\
%\phi_2 (\bm{x})&=x_2 + log_{e}100\\
%\bm{w}&=(1,-1)\\
%f(\bm{x})&=x_1 - x_ 2 - log_{e}100\\
%\end{aligned}
%\end{equation*}
\textbf{(d)}
Yes\\
\begin{equation*}
\begin{aligned}
\phi_1 (\bm{x})&=x_1 x_2\\
\bm{w}&=-1\\
f(\bm{x})&=- x_1 x_ 2\\
\end{aligned}
\end{equation*}
\section*{Problem 3}
\begin{center}
Linear support vector machine\\
\begin{tabular}{|c|c|c|c|}
		\hline
		\multicolumn{2}{|c|}{}
		&\multicolumn{2}{|c|}{Predicted}\\
		\cline{3-4}
		\multicolumn{2}{|c|}{}& +  & -  \\
		\cline{1-4}
		Actual & +         & 2473 & 222 \\
		\cline{2-4}
		       & -         & 315 & 1591\\
		\hline
\end{tabular}\\
\end{center}
\begin{equation*}
\begin{aligned}
 Error\ Rate&=\frac{222+315}{4601}=0.1167\\
\end{aligned}
\end{equation*}
\textbf{(b)}
\begin{center}
Nonlinear support vector machine\\
	\begin{tabular}{|c|c|c|c|}
		\hline
		\multicolumn{2}{|c|}{}
		&\multicolumn{2}{|c|}{Predicted}\\
		\cline{3-4}
		\multicolumn{2}{|c|}{}& +  & -  \\
		\cline{1-4}
		Actual & +         & 2542 & 191 \\
		\cline{2-4}
		& -         & 174 & 1694 \\
		\hline
	\end{tabular}\\
\end{center}
\begin{equation*}
\begin{aligned}
	Error\ Rate&=\frac{191+174}{4601}=0.07933\\
\end{aligned}
\end{equation*}
Nonlinear support vector machine is more effective, because it has a smaller error rate.

\begin{lstlisting}[keywordstyle=\color{blue!70},commentstyle=\color{red!50!green!50!blue!50},frame=shadowbox, rulesepcolor=\color{red!20!green!20!blue!20}] 
% NAN CAO CSE881 HW5
clear;
% set dir in nan's win lap
% cd C:\Users\nan66\Dropbox\CSE881\HW5\;
% set dir in nan's linux lap
cd /home/nan/Dropbox/CSE881/HW5;
% set dir in remote server
% cd /CSE881/HW5;
A = load('spambase.data');
N = size(A, 1);
seed = 52871775; % seed for random number generator
rng(seed); % for repeatability of your experiment
A = A(randperm(N),:); % this will reshuffle the rows in matrix A
X=A(:,1:57);
Y=A(:,58);
%You need to compare the performance of the linear and non-linear support
%vector machine classifiers on this data set using nested cross-validation.
%Use k=10 for the outer loop (classifier evaluation) and k=5 for the inner
%loop (model selection) of the nested cross-validation procedure
N=length(A);
indices = crossvalind('Kfold',N,10);%outer loop
%%%%%%%%%%%%%%%%%%%%%%%%%%%%%%%%%%%%%%
%Linear
%%%%%%%%%%%%%%%%%%%%%%%%%%%%%%%%%%%%%%
lambda = logspace(-4,3,11); % create a set of candidate lambda values
cm3a=zeros(2,2); %create a bin to store the confusion matrix
bestLambda=0;%initialize the best lambda
for k=1:10
Lam=0
testID = (indices == k); %id of test data in kth round
trainID = ~testID; %id of train data in kth round
Xtrain = X(trainID,:);
Ytrain = Y(trainID);
Xtest = X(testID,:);
Ytest = Y(testID);
% tune the hyper-parameter lambda for linear SVM.
model = fitclinear(Xtrain, Ytrain, 'Kfold', 5, 'Learner', 'svm', 'Lambda', lambda);
foldNumber = 3; % to examine the model created for the 3rd fold
model.Trained{foldNumber}
ce = kfoldLoss(model) % to examine the classification error for each lambda
Lam=lambda(ce==min(ce));
bestLambda(k)=Lam(1);
SVMmodel = fitclinear(Xtrain, Ytrain, 'Learner', 'svm', 'Lambda', bestLambda(k));
pred = predict(SVMmodel, Xtest);
cp = classperf(Ytest);
classperf(cp,pred);
cp.DiagnosticTable % to show the confusion matrix
cm3a=cm3a+cp.DiagnosticTable;% sum the confusion matrix
cp.ErrorRate % to show the classification error
end;
cm3a
errorrate3a=trace(rot90(cm3a))/sum(cm3a(:))
%%%%%%%%%%%%%%%%%%%%%%%%%%%%%%%%%%%%%%
%Non-Linear
%%%%%%%%%%%%%%%%%%%%%%%%%%%%%%%%%%%%%%
%%%%%---------------------------------------------------------------------------------------
% for testing use only
%dlmwrite(Lam3b.txt,bestLambda); 
%dlmwrite(Sig3b.txt,bestSigma); 
%dlmwrite(indices.txt,indices);
%dlmwrite(A.txt,A);
%%%%%%%%%%%%%%%%%%%%%%%%
% indices=load('indices.txt')
% bestLambda=load('Lam3b.txt')
% bestSigma=load('Sig3b.txt')
%%%%%--------------------------------------------------------------------------------------
lambda = logspace(-3,3,11); % create a set of lambda values
sigma = logspace(-3,3,11); % create a set of kernel scale values
cvloss = zeros(11,11); % stores the CV error for each (lambda,sigma) pair
cm3b=zeros(2,2); %create a bin to store the confusion matrix
bestLambda=0;%initialize the bestlamda
for k=1:10
testID = (indices == k); %id of test data in kth round
trainID = ~testID; %id of train data in kth round
Xtrain = X(trainID,:);
Ytrain = Y(trainID);
Xtest = X(testID,:);
Ytest = Y(testID);
for i=1:11
for j=1:11
SVMmodel = fitcsvm(Xtrain, Ytrain, 'KernelFunction','RBF','KernelScale',sigma(j),'BoxConstraint',lambda(i),'Kfold', 5);
cvloss(i,j)=kfoldLoss(SVMmodel);
end;
end;
[a,b]=find(cvloss==min(cvloss(:)));
i1=a(1);% just in case if there are two or more min values
j1=b(1);
bestLambda(k)=lambda(i1);
bestSigma(k)=sigma(j1);
SVMmodel = fitcsvm(Xtrain, Ytrain,'KernelFunction','RBF','KernelScale',bestSigma(k),'BoxConstraint',bestLambda(k));
pred = predict(SVMmodel, Xtest);
cp = classperf(Ytest);
classperf(cp,pred);
cp.DiagnosticTable % to show the confusion matrix
cm3b=cm3b+cp.DiagnosticTable;% sum the confusion matrix
cp.ErrorRate % to show the classification error
end;
cm3b % confusion matrix
errorrate3b=trace(rot90(cm3b))/sum(cm3b(:)) % errorrate
\end{lstlisting} 
\end{document}