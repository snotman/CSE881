\documentclass[11pt]{scrartcl}
\usepackage[utf8]{inputenc}
\usepackage{mathtools}
\usepackage{amssymb}
\usepackage{listings}
\lstset
{ %Formatting for code in appendix
	language=Matlab,
	basicstyle=\footnotesize,
	numbers=left,
	stepnumber=1,
	showstringspaces=false,
	tabsize=1,
	breaklines=true,
	breakatwhitespace=false,
}
\begin{document}
\centerline{\LARGE{\textbf{CSE881 HW1}}}
\centerline{\large{\textit{Nan Cao,\  A52871775}}}
\centerline{\large{\textit{Sep 9th, 2016}}}
% \maketitle
\section*{Problem 1}
\textbf{(a)}: Discrete; Qualitative; Nominal;\\
\textbf{(b)}: Discrete; Qualitative; Ordinal;\\ 
\textbf{(c)}: Continuous; Quantitative; Ratio;\\
\textbf{(d)}: Continuous; Quantitative; Interval;\\
\textbf{(e)}: Continuous; Quantitative; Interval;\\
\textbf{(f)}: Discrete; Qualitative; Ordinal\\
\section*{Problem 2}
\textbf{(a)}: Yes, because salary of computer engineers is an ordinal attribute, and median works good on it.\\
\textbf{(b)}: Yes, because weight and height of individuals are ratio attribute, and correlation is proper to ratio attributes.\\ 
\textbf{(c)}: No, because Richter Magnitude Scale is an ordinal attribute, and it's a logarithmic scale instead of a linear one. \\
\textbf{(d)}: Yes, because gender is a nominal attribute, and entropy is proper to this kind of data.\\
\textbf{(e)}: No, because degree temperature in Degree Celsius is an interval attribute, geometric doesn't work on it. \\
\textbf{(f)}: No, because GPA is an ordinal attribute, standard deviation doesn't work on it.\\
\section*{Problem 3}
\textbf{(You can find my source code and outputs at the end of the homework.)}
\begin{equation*}
\begin{aligned}
y_x&=(0.2901,  0.2574, 0.7762)\\
w&=(-0.7960, 0.5970,0.0995)
\end{aligned}
\end{equation*}
\section*{Problem 4}
\textbf{(a)}:\\
\begin{equation*}
\begin{aligned}
\mathbf{\rho}
&=
\begin{bmatrix} 
1 & \frac{199.37}{\sqrt{389.75*610.52}} & \frac{135.12}{\sqrt{389.75*359.36}} \\ 
\frac{199.37}{\sqrt{389.75*610.52}}  & 1 & \frac{426.30}{610.52*359.36} \\ 
 \frac{135.12}{\sqrt{389.75*359.36}}  & \frac{426.30}{610.52*359.36}  &1 \\
\end{bmatrix} \\
&=
\begin{bmatrix} 
1 &  0.4087 &  0.3610 \\ 
0.4087 & 1 & 09101  \\ 
 0.3610 & 0.9101  &1 \\
\end{bmatrix} 
\end{aligned}
\end{equation*}
Yes, according to the correlation matrix, weight is more correlated to age.\\
\textbf{(b)}:\\
No, the covariance will remain the same. 
\begin{equation*}
\begin{aligned}
A&=age;\  
W=weight\\
W'&=centered\ weight=W-\bar{W}=W-E(W);\\
E(W')&=E(W-\bar{W})=0\\
Cov(A,W')&=E(A-E(A))E(W'-E(W'))\\
&=E(A-E(A))E(W'-0)\\
&=E(A-E(A))E(W-E(W))\\
&=Cov(A,W)\\
\end{aligned}
\end{equation*}
\textbf{(c)}:\\
Covariance between weight (in kilogram) and age be smaller than 199.37.
\begin{equation*}
\begin{aligned}
A&=age\\
W_p&=weight \ (in\ pounds)\\
W_k&=weight \ (in\ kilograms)\\
W_p&=2.2W_k; \  W_k=\frac{5}{11} W_p\\
Cov (A,W_k)
&=Cov(A,\frac{5}{11} W_p)\\
&=E(A-E(A))E(\frac{5}{11} W_p)-E(\frac{5}{11} W_p)))\\
&=\frac{5}{11} E(A-E(A))E( W_p-E(W_p))\\
&=\frac{5}{11} Cov(A,W_p)\\
&=\frac{5}{11} *199.37\\
&=54.2591 < 119.37\\
\end{aligned}
\end{equation*}
\textbf{(d)}:\\
Their covariance value be  199.37
\begin{equation*}
\begin{aligned}
A&=age;\   
W=weight\\
A^*&=\frac{A- \bar{A}}{\sigma_A}\\
W^*&=\frac{W- \bar{W}}{\sigma_W}\\
Cov(A^*,W^*)
&=E(A^*-E(A^*))E(W^*-E(W^*))\\
&=E(\frac{A- \bar{A}}{\sigma_A}-E(\frac{A- \bar{A}}{\sigma_A}))E(\frac{W- \bar{W}}{\sigma_W}-E(\frac{W- \bar{W}}{\sigma_W}))\\
&=\frac{E(A-E(A))E(W-E(W))}{\sigma_A \sigma_W}\\
&=\frac{Cov(A,W))}{\sigma_A \sigma_W}
=\frac{199.37}{389.75 *610.52}<199.37
\end{aligned}
\end{equation*}
\section*{Problem 5}
\begin{equation*}
\begin{aligned}
p
&=\frac{2NQ+Z_{\frac{\alpha}{2}}^{2}
		\pm Z_{\frac{\alpha}{2}}
		\sqrt
			{
			Z_{\frac{\alpha}{2}}^{2}+4NQ-4NQ^2
			}
	}
	{2(N+Z_{\frac{\alpha}{2}}^{2})}\\
&=\frac{2*150*0.95+1.96^2
		\pm 1.96
		\sqrt
			{
			1.96^2+4*150*0.95-4*150*1.96^2
			}
	}
	{2(150+1.96^2)}\\
&=(0.90254,0.97499)\\
\end{aligned}
\end{equation*}
Yes, it is safe to conclude (say, with 95\% confidence) that my method outperforms the baseline method.
\section*{Source Codes \& Output for Problem 3}
\textbf{Source Codes}
\begin{lstlisting}
X=[0.1,0.1,0.2;0.5,0.6,0.4]'
y=[0.1,0.4,0.8]'
[U S V]=svd(X)
A=U(1:3,1:2)
P=A*(A'*A)^(-1)*A'
yx=P*y
w=cross(X(1:3,1),X(1:3,2))
w=w/norm(w)
\end{lstlisting}
\textbf{Outputs}
\begin{lstlisting}
X =

    0.1000    0.5000
    0.1000    0.6000
    0.2000    0.4000


y =

    0.1000
    0.4000
    0.8000


U =

   -0.5632   -0.2216   -0.7960
   -0.6705   -0.4404    0.5970
   -0.4829    0.8700    0.0995


S =

    0.9042         0
         0    0.1111
         0         0


V =

   -0.2432    0.9700
   -0.9700   -0.2432


A =

   -0.5632   -0.2216
   -0.6705   -0.4404
   -0.4829    0.8700


P =

    0.3663    0.4752    0.0792
    0.4752    0.6436   -0.0594
    0.0792   -0.0594    0.9901


yx =

    0.2901
    0.2574
    0.7762


w =

   -0.0800
    0.0600
    0.0100


w =

   -0.7960
    0.5970
    0.0995
\end{lstlisting}

\end{document}
